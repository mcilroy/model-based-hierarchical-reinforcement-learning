\documentclass{article} % For LaTeX2e
\usepackage{nips15submit_e,times}
\usepackage{hyperref}
\usepackage{url}
\usepackage{graphicx}
\usepackage{caption}
\usepackage{subcaption}

%\documentstyle[nips14submit_09,times,art10]{article} % For LaTeX 2.09


\title{Model-Based Hierarchical Reinforcement Learning}


\author{
Stuart McIlroy \\
Department of Computer Science\\
Dalhousie University\\
Halifax, Nova Scotia, Canada \\
\texttt{st870547@dal.ca} \\
}

% The \author macro works with any number of authors. There are two commands
% used to separate the names and addresses of multiple authors: \And and \AND.
%
% Using \And between authors leaves it to \LaTeX{} to determine where to break
% the lines. Using \AND forces a linebreak at that point. So, if \LaTeX{}
% puts 3 of 4 authors names on the first line, and the last on the second
% line, try using \AND instead of \And before the third author name.

\newcommand{\fix}{\marginpar{FIX}}
\newcommand{\new}{\marginpar{NEW}}

\nipsfinalcopy % Uncomment for camera-ready version

\begin{document}

\graphicspath{{figures/}}
\maketitle

\begin{abstract}
Reinforcement learning (RL) techniques have recently shown improvement over existing approaches to some problems such as game playing. However, reinforcement learning models still suffer from having to scale to large state spaces. Hierarchical reinforcement learning has been shown to aid in scaling RL techniques by abstracting actions. We propose a model-based hierarchical reinforcement learning approach using monte carlo tree search. We found that the overall number of training trials is reduced. This finding illustrates the advantages of model-based learning methods.
\end{abstract}


\section{Introduction}

Reinforcement learning has recently shown strong results in blah blah...~\cite{mnih2015human}.

\subsection{Hierarchical Reinforcement Learning}

Hierarchical reinforcement learning has done other things...

\subsubsection{Model-based}

Model based ...




%\begin{figure}[h]
%\begin{center}
%%\framebox[4.0in]{$\;$}
%\fbox{\rule[-.5cm]{0cm}{4cm} \rule[-.5cm]{4cm}{0cm}}
%\end{center}
%\caption{Sample figure caption.}
%\end{figure}



%\begin{table}[t]
%\caption{Sample table title}
%\label{sample-table}
%\begin{center}
%\begin{tabular}{ll}
%\multicolumn{1}{c}{\bf PART}  &\multicolumn{1}{c}{\bf DESCRIPTION}
%\\ \hline \\
%Dendrite         &Input terminal \\
%Axon             &Output terminal \\
%Soma             &Cell body (contains cell nucleus) \\
%\end{tabular}
%\end{center}
%\end{table}

\bibliography{mybib}{}
\bibliographystyle{plain}
\end{document}
